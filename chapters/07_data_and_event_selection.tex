\chapter{Data Sample and Event Selection}\label{ch:data_and_event_selection}

\section{Event Selection} \label{sec:event_selection}
This analysis uses the entirety of the 2015 and 2016 ATLAS datasets, comprising \linebreak
$36.1~fb^{-1}~(\pm2.1\%)$ of integrated luminosity.
All collisions occurred as $\sqrt{s}=13~TeV$.

The trigger selects events with $H_{T}>1.0~TeV$, where $H_{T}$ is the scalar sum of the $p_{T}$ of all jets in an event.
The trigger efficiency for the benchmark signal samples is found to be $100\%$.

Additionally, events are required to have a primary vertex formed from at least two tracks with $p_{T}>400~MeV$ each.

Two different methods of reconstructing jets are used in the analysis.
Large-R jets are reconstructed using the anti-$k_{T}$ algorithm with radius parameter $R=1.0$,
and sub-jet radius parameter $R_{sub-jet}=0.2$.
The jets are trimmed with sub-jet $p_{T}$-fraction $f_{p_{T}}=0.05$.
The details of how jets are reconstructed, trimmed, and calibrated can be found in chapter~\ref{ch:jets}.

Small-R jets are reconstructed with the anti-$k_{T}$ algorithm, using radius parameter $R=0.4$.

Large-R jets are required to have $p_{T}>200~GeV$ and $|\eta|<2.0$.
The leading large-R jet is required to have $p_{T}>440~GeV$, a selection for which the $H_{T}$ trigger is fully efficient.

Small-R jets are used as candidates for b-tagging.
To be considered for b-tagging, a small-R jet must have $p_{T}>50~GeV$ and $|\eta|<2.5$.
The algorithm used to tag b-jets is described in chapter~\ref{ch:jets} as well as~\cite{b-jet-perf-1,b-jet-perf-2}.
The algorithm used for b-tagging in this analysis is the fixed-efficiency $70\%$ working point.

The control, validation, and signal regions defined in section~\ref{sec:region_defs} can be further segmented into b-tag and b-veto regions.
Events with at least one b-tagged small-R jet are considered b-tag events, while those without are labelled as b-veto.
When b-tagging is not taken into consideration, the region is called b-inclusive.

Additionally, large-R jets found to be within $\Delta R=1.0$ of a b-tagged small-R jet in the same event are referred to as b-matched jets.
Those large-R jets not within $\Delta R=1.0$ of a b-tagged small-R jet are called non-b-matched.

The distinction in terminology between b-tagged and b-matched is important enough to restate for clarity.
When referring to the presence or absence of a b-tagged jet \textit{within an event}, the terms b-tag, b-veto, and b-inclusive are used.
When referring to the proximity of an \textit{individual large-R jet} to a b-tagged small-R jet, the terms b-matched and non-b-matched are used.

Templates will be divided into b-matched and non-b-matched templates, which are used to dress b-matched and non-b-matched jets, respectively.
Validation and signal regions will be divided into b-tag, b-veto, and b-inclusive regions.