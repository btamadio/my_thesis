\chapter{Jets} \label{ch:jets}

A jet is a collimated spray of stable particles measurable in the detector,
which arises from the production of a quark or gluon in the scattering process.
Due to QCD confinement, the quarks and gluons themselves can never be observed,
so jets serve as a kind of observable proxy for these more fundamental particles.
However, there is not a simple one-to-one correspondence between each jet measured in an event and a quark or gluon in the hard scattering process.
Due to the complex nature of proton-proton collisions, as well as the fundamental probabilistic nature of quantum mechanics,
the partonic source for a given jet can never be specified exactly.
Likewise, when performing theoretical calculations, the number and property of jets that arise from a given parton produced in the hard scattering
can only be specified probabilistically.
Section~\ref{sec:jet_collisions} will explain the many different parts of a proton-proton collision that contribute to jet production.

Furthermore, the number and properties of jets measured in a single event will depend on the choice of jet reconstruction algorithm.
In order for experimentalists to test the predictions of theory, it's important for standard jet definitions to be decided upon.
While there is no one theoretically correct choice of jet definition, there are certain properties of jet algorithms
that make them more or less desirable.
The different kinds of jet algorithms and their properties will be discussed in section~\ref{sec:jet_clustering}

Section~\ref{sec:jet_substructure} will discuss jet substructure, which are the internal properties of jets.
And section~\ref{sec:jet_reconstruction} will explain the process of jet reconstruction and calibration in ATLAS.

\section{The proton-proton collision environment}\label{sec:jet_collisions}

As discussed in~\ref{subsec:qcd_pdfs}, the goal of proton-proton collision experiments is to understand the interactions
between fundamental particles, but due to QCD confinement it is impossible to simply collide the individual partons of interest.
Instead, complicated bound states of quarks and gluons are collided,
with the goal of measuring the processes that occur when constituent partons collide with each other.
As a result, proton-proton collision events create a very messy environment from which the hard-scatter process
has to be deduced.

QCD confinement also has the consequence that the final state quarks and gluons created in LHC collisions
can never be directly detected.
Instead, collimated sprays of hadrons, called jets, must be measured in order to attempt to reconstruct the
hard scattering process that gave rise to them.

Figure~\ref{fig:jet_tth_diagram} illustrates a typical proton-proton collision event,
in which two gluons annihilate to generate a Higgs boson along with a top-antitop quark pair.
Even though this is not a dijet or multijet production event,
it is useful for understanding all the parts of a proton-proton collision that must be considered when calculating the
predicted rates of multijet production.

\begin{figure}[h!]
    \centering
\includegraphics[width=0.6\linewidth]{jet_tth_diagram}
\caption{Diagram of a proton-proton collision event in which the hard scatter process is $t\bar{t}H$ production.
Illustrated are the initial state radiation, underlying event, hard-scatter process, final state parton shower,
fragmentation, hadron decays, and final state QED radiation.}
\label{fig:jet_tth_diagram}
\end{figure}\cite{sherpa-2004}

The incoming protons are illustrated by green blob with three incoming arrows to represent the constituent quarks.
The initial state parton showering, governed by QCD, is shown in blue.
The hard scatter process, in which two gluons annihilate to produce a Higgs boson and top-antitop pair is represented
by the large red circle.
Quarks and gluons from the incoming proton that do not participate in the hard scatter process can nonetheless interact
with each other, creating a so-called underlying event, shown in purple.
The Higgs decays to a quark-antiquark pair, shown in red, and all of the strongly-interacting final state quarks
and gluons undergo final state parton showering, also in red.
Once the final state parton shower particles reach a low enough energy, they hadronize, a non-perturbative process
indicated by the light green blobs.
The resulting hadrons then decay through various decay chains, shown in dark green.
Photon radiation, governed by QED, can occur at any of these stages, and is shown in yellow.

The final measured objects in the detector are the jets.
But every process leading up to the final state is quantum mechanical,
meaning that interference terms between the various processes lead to a fundamental and unresolvable ambiguity in the source of a given jet.

An event with 3 high-$pT$ jets could be explained by a $2\rightarrow3$ hard-scattering process,
or a $2\rightarrow2$ hard scattering process, with an additional jet arising from ISR or FSR, the underlying event,
or even from a hadronic decay.
All of these processes must be taken into account when calculating the rate of 3-jet events,
each contributing its own source of uncertainty.
The final rate calculation combines the perturbative QCD calculations for the matrix elements and parton showering
with empirically measured probability distributions for the non-perturbative parts of the collision,
including soft gluon emission, long-distance couplings, hadronization, and multi-parton interactions (MPI). 
As the number of jets in the final state increases, so does the complexity and uncertainty when calculating event rates.

Monte Carlo (MC) generators attempt to account for all of these effects to calculate the rate of multijet events at the LHC .
However, the MC estimates have very large uncertainties, for the reasons given above.
As a result, data driven estimation methods are used for determining the background rate of multijet events.

\section{Clustering Algorithms}\label{sec:jet_clustering}

In order to measure jets in an event, a decision has to be made on the type of jet reconstruction algorithm to use.
The observable result of a collision event can consists of many clusters of stable particles or calorimeter hits.
Exactly which particles to cluster together is highly non-trivial, and developing the algorithms to make these decisions
is an active area of research in particle physics.

The 1990 Snowmass accord defined a set of criteria which should be met by any jet algorithm.
The criteria for a developing a jet reconstruction algorithm are:

\begin{enumerate}
    \item Simple to implement in an experimental analysis
    \item Simple to implement in the theoretical calculation
    \item Defined at any order in perturbation theory
    \item Yields finite cross sections at any order of perturbation theory
    \item Yields a cross section that is relatively insensitive to hadronization
\end{enumerate}\cite{jet-jetography,jet-snowmass}

Jet reconstruction algorithms that meet all of these criteria allow experimentalists to test theoretical prediction,
because both the theoretical calculations and experimental analysis can use the same jet definition.
Jet algorithms are defined as acting on abstract "objects".
The last Snowmass criteria ensures that in theoretical calculations,
jet algorithms can be applied at either the parton-level or hadron level.
In either case, the object which the algorithm acts on will be a particle.
Jet algorithms can also be applied by experimentalists to reconstruct jets from either simulated or real calorimeter hits.
In this case, the object which the algorithm acts on is a cluster of calorimeter hits.

Historically, there have been two major types of jet algorithms used in particle physics experiments:
cone-based algorithms and sequential recombination algorithms.
Cone-based algorithms were developed first.
They typically use the highest-transverse-momentum (hardest) object in the event as a seed for a jet cone,
then define a cone around that seed as the leading jet,
before moving on to the next-hardest object.
Cone-based algorithms have the advantage of producing regularly-shaped jet boundaries,
which can simplify theoretical calculations and make experimental calibration easier~\cite{jet-cone-algo}.
However, cone-based algorithms typically collinear safety,
which are properties that need to be satisfied to meet the Snowmass criteria.

\subsection{IRC Safety}\label{subsec:jet_irc_safety}

Jet algorithms should ideally possess infrared and collinear (IRC) safety.
This means that their results should not change due to soft gluon emissions or near-collinear emissions
during either the parton shower or hadronization process.
IRC safety is important because soft and collinear emissions are extremely common in QCD,
occur randomly, are hard to predict, and can lead to divergences in perturbative calculations if the algorithm is IRC-unsafe.\cite{jet-jetography}

\subsubsection{Collinear safety}

A collinear-safe algorithm is insensitive to near-collinear emissions that occur during parton showering or hadron decays.
Cone-based algorithms are collinearly unsafe because they seed jets with the hardest object in the event,
which is very likely to change if there is a near-collinear emission.

In figure~\ref{fig:jet_collinear_safety}, for the collinear-unsafe cone-based algorithm,
the hardest object in the event changes depending on whether the radiated gluon is reabsorbed or emitted nearly collinear to the original quark.
Thus, the resulting seed will change, yielding different jet results.
The loop diagram and collinear emission diagram both lead to divergences in the theoretical cross-section.
In a collinear-safe algorithm, the two divergences would cancel out because both amplitudes must be added to calculate the rate
of the single jet being produced.
But in a collinear-unsafe algorithm, the two divergences do not cancel out, so perturbation cross-sections are not finite.

\begin{figure}[h!]
    \centering
\includegraphics[width=0.9\linewidth]{jet_collinear_safety}
\caption{Illustration comparing the results of collinear-safe (left) and collinear-unsafe (right) jet clustering algorithms.
The emission of a near-collinear gluon changes the number of jets in the event, and results in diverges in theoretical calculations.
The x-axis represents rapidity, and y-axis represents transverse momentum.}
\label{fig:jet_collinear_safety}
\end{figure}\cite{jet-jetography}

\subsubsection{Infrared safety}

A related concept is that of infrared safety.
Jet algorithms should insensitive to soft gluon emissions.
These emissions are extremely common during parton showering, and since they occur in the non-perturbative regime of QCD,
very hard to accurately predict.

Figure~\ref{fig:jet_ir_safety} illustrates the result of an infrared-unsafe algorithm.
A W boson decays to a quark-antiquark pair, which should result in two hard jets.
Diagrams (b) and (c) each lead to IR divergences, which would cancel if the algorithm is infrared-safe.
But for an infrared-unsafe algorithm, diagram (c) leads to a different number of jets from diagram (b),
so the divergences do not cancel.

\begin{figure}[h!]
    \centering
\includegraphics[width=0.9\linewidth]{jet_ir_safety}
\caption{Illustration of infrared unsafety.
In an event with a W boson decaying to two hard partons, the emission of a soft gluon can change the result of an infrared-unsafe algorithm.}
\label{fig:jet_ir_safety}
\end{figure}\cite{jet-jetography}

\subsection{Sequential recombination}\label{subec:jet_seq_recombination}

Sequential recombination algorithms are the most commonly used jet algorithms in ATLAS today,
and consist of the $k_T$ algorithm\cite{jet-kt-algo}, Cambridge-Aachen (C/A)\cite{jet-ca-algo}, and anti-$k_T$ algorithm\cite{jet-antikt-algo}.
All sequential recombination algorithms start by defining a distance measure.


\subsubsection{Cambridge-Aachen (p=0)}

\subsubsection{$k_T$ (p=1)}

\subsubsection{anti-$k_T$ (p=-1)}

\section{Substructure}\label{sec:jet_substructure}
\subsection{Boosted objects}\label{subsec:jet_boosted_substructure}
\subsection{Accidental substructure}\label{subsec:jet_accidental_substructure}

\section{Reconstruction in ATLAS}\label{sec:jet_reconstruction}
\subsection{Clustering}\label{subsec:jet_clustering}
\subsection{Calibration}\label{subsec:jet_calibration}
\subsection{Full and Fast Simulation Comparison}\label{subsec:jet_full_vs_fast_sim}
\subsection{Flavor Tagging}\label{subsec:jet_flavor_tagging}