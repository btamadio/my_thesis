\chapter{Signal Modelling and Systematics} \label{ch:signal}

\section{Signal Modeling}
% TODO: Explain MC generators used, include grid of mG/mX used, number
% of events simulated, etc.
% Might need to find a backup note if it exists

\section{Signal Systematic Uncertainties}
There are four main components to the systematic uncertainty on the
estimated signal yield. These components are the large-R jet mass
scale (JMS), the small-R b-tagging uncertainty, the Monte Carlo statistical
uncertainty, and the modelling uncertainty, which includes uncertainty
on parton distribution functions (PDFs), QCD scale uncertainty, and
initial state radiation (ISR) modelling uncertainty.

A given systematic is evaluated by varying a nuisance parameter and
determining the percent difference in signal yield between the nominal
distribution and systematically varied distribution. When a systematic
contains multiple components, those components are treated as
uncorrelated, and their contributions are combined in quadrature.

Systematic uncertainties depend on the gluino and neutralino masses
being considered, as well as the decay mode of direct or cascade. As
such, they are evaluated separately at each point in the mass grid.

\subsection{Jet Mass Scale Uncertainty}
For the JMS uncertainty, there are four components, called the
baseline, modeling, statistical, and tracking components. These
components are derived from the $R_{trk}$ method
\cite{rtrk-method}. The JMS uncertainty is largest for
$m_{\tilde{g}}=1.0~TeV$, at $\approx 24\%$, and drops to $\approx8\%$
for signal points with $m_{\tilde{g}}=1.8~TeV$. It is generally
dominated by the tracking uncertainty, followed by the baseline uncertainty.

\subsection{b-Tagging Uncertainty}
The b-tagging uncertainty is evaluated by varying a set of 25
nuisance parameters. The result is an uncertainty on the signal
efficiency of between $15\%$ and $25\%$. This uncertainty is only
applied to the b-tag signal regions.

\subsection{Monte Carlo Statistical Uncertainty}
The Monte Carlo statistical uncertainty accounts for the fact that
only a limited sample of simulated events are produced for each mass
point. The uncertainty is derived as
$\sigma=\sqrt{\epsilon(1-\epsilon)/N}$, where $\epsilon$ is the
efficiency measured in the sample, and $N$ is the sample size.


\subsection{PDF, QCD scale, and ISR Uncertainties}
Evalulating the contribution to the signal efficiency uncertainty from
PDF, $\alpha_s$ and ISR modelling uncertainties requires the
generation of truth-level signal simulation samples where different
parameters are varied in the generation.

For PDF uncertainties, the internal event weights in the PDF set are varied up and
down during the generation. 

For the QCD scale uncertainty, the value of $alpha_s$ is varied up and
down durin the generation.

For ISR uncertainties, the value of the matching scale,
$q_{\textrm{cut}}$ is varied up and down during the generation.

The PDF and QCD scale contributions to the uncertainty are highest at
low gluino mass, reaching a maximum of $25\%$ at
$m_{\tilde{g}}=1.0~TeV$, the lowest gluino mass studied. For higher
masses, these uncertainties drop to only a few percent.

\subsection{Summary}
The various signal systematic uncertainties are summarized in table
\ref{tbl:signal_systs}
