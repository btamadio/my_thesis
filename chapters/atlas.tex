\chapter{The ATLAS Detector}

\section{Detector Overview}
ATLAS is a system of particle detectors built to
measure collisions of both proton-proton and heavy ion collisions
generated by the LHC. \cite{atlas-detector-2008} The entire detector
is $44~m$ long and $25~m$ in diameter. 

The innermost detector consists
of a pixel detector and silicon strip tracker, both used to measure
the tracks of charged particles. The next layer is the transition
radiation tracker, used for both tracking and electron identification.

The pixel detector, silicon strip detector, and transition radiation
tracker are together known as the inner detector, and are all
contained inside a $2~T$

\section{Coordinates}
The coordinate system used in this document is the standard ATLAS
coordinate system, which is defined in this section. For both Cartesian and
polar coordinates, the origin is defined as the nominal interaction
point. The z-axis points down the beamline. The x-y plane is
perpendicular to the beamline. The positive y-direction points up, and
the positive x-direction points towads the center of the LHC ring. The
positive z-direction therefore points counterclockwise along the LHC,
when view from above, as required by a right-handed coordinate system.

For cylindrical coordinates, the z-axis is defined the same way as for
Cartesian coordinates. The azimuthal angle $\phi$ is the angle
from the positive x-axis, while the polar angle $\theta$ is the angle
from the beamline.

A more convenient measure of angle from the beamline is the
\textit{rapidity}, because rapidity differences are invariant under Lorentz boosts in the
z-direction. The rapidity is defined as:
\begin{equation}
y = \frac{1}{2}\ln\frac{E+p_z}{E-p_z}
\end{equation}

Another frequently used quantity is the
\texit{pseudorapidity}, which is defined as:
\begin{equation}
\eta = -\ln\frac{\theta}{2}
\end{equation}

Pseudorapidity differences are also invariant with respect to
longitudinal Lorentz boosts. In the limit where $p_T << m$, rapidity and
pseudorapidity are equal.

Pseudorapidity ranges from $\eta=0$, which is perpendicular to the
beamline, up to $\eta=\inf$, which is parallel to the beamline.

Finally, a distance measure in $\eta-\phi$ space is often used. This
distance, $\DeltaR$ is defined as:

\begin{equation}
\DeltaR = \sqrt{\Delta\eta^2+\Delta\phi^2}
\end{equation}


\section{Magnet System}
\section{Inner Detector}
\subsection{Pixels}
\subsection{Silcon Tracker}
\subsection{Transition Radiation Tracker}
\section{Calorimeters}
\subsection{Electromagnetic Calorimeters}
\subsection{Hadronic Calorimeters}
\section{Muon Spectrometer}
\section{Trigger and Data Acquisition}