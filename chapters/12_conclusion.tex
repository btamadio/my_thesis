\chapter{Conclusion}\label{ch:conclusion}

A search was performed for supersymmetric particles called gluinos decaying through R-parity-violating vertices to high sum-of-jet-mass final states.
The search was based on $36.1~fb^{-1}$ of $\sqrt{s}=13~TeVe$ proton-proton collision data taken by the ATLAS detector at the LHC in 2015 and 2016.
There were two signal decay processes under consideration, one in which the gluinos each decay to three Standard Model quarks, and one in which the gluino decays to a neutralino and two quarks, and the neutralino decays to three quarks.
Either of those processes would result to an excess in the number of events with a large sum of leading jet masses over the Standard Model background.

A data-driven jet mass template method was used to estimate the contribution from the Standard Model backgrounds.
New features of this analysis are using a data-driven estimate of background statistical uncertainty derived from the jet mass templates, and using b-tagging to boost the expected signal-to-noise ratio.
No significant excess above the background prediction was found in any of the signal regions.
Based on the observed yields, limits were placed on the production of gluinos decaying through either the direct decay or cascade decay scenarios.

For the direct decay mode, $95\%~CL$ upper bounds on the gluino production cross section range from $0.8~fb$ for a gluino with mass of $900~GeV$, to $0.011~fb$ for a gluino with mass $1.8~TeV$.
These upper bounds are above the theoretical cross-section for gluinos in this mass range, so the search failed to exclude gluinos decaying via this process.
For the cascade decay mode, a range of gluino and neutralino mass points could be excluded based on the cross-section upper bounds.
Depending on the neutralino mass, gluinos with masses between $1~TeV$ and $1.8~TeV$ can be excluded, with the maximum gluino mass excluded for neutralino mass around $1~TeV$ .
This search has provided the strongest limits yet for the R-parity-violating UDD decays of gluinos, and significantly extends the reach from the Run-1 analysis.
