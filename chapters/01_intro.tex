\chapter{Introduction} \label{ch:intro}

Experimental particle physics is the systematic probing of the fundamental constituents of matter,
in order to uncover new insights into laws of nature applicable at the smallest distance scales.
To date, the most successful and predictive theory explaining the interactions of elementary particles is the Standard Model.
The Standard Model accounts for all of the known elementary particles and their interactions, with the notable exception of gravitational interactions.

In July 2012, the final piece in Standard Model particle puzzle, the Higgs boson, was discovered by the CMS and ATLAS experiments at the
Large Hadron Collider.
This was the last particle predicted by the Standard Model that had yet to be discovered.
But there are still many questions left unanswered by the Standard Model.
Answering these questions will require a new theory.
This theory must be consistent with all the experimentally-confirmed predictions of the Standard Model,
but also go further to explain what the Standard Model can't.
Such theories are known as Beyond-the-Standard-Model (BSM) theories.

One of the most promising BSM theories is known as Supersymmetry, or SUSY, which posits a new symmetry of nature.
This new symmetry connects the two types of fundamental fields - fermions and bosons.
SUSY provides an elegant solution to many of the mysteries of the Standard Model.
However, it also predicts a slew of new fundamental particles, no evidence of which has yet to be observed.

The ATLAS detector at the Large Hadron Collider was built with a wide-ranging set of physics goals.
In addition to the discovery of the Higgs, the LHC and ATLAS were built to search for evidence of Supersymmetric particles.
SUSY has a large number free parameters, whose values can't be predicted from first principles, but would have to be measured experimentally.
A SUSY search that fails to discover a new particle nonetheless will often result in constraining the viable parameter space for the theory.

This thesis focuses on a search for a proposed SUSY particle called the gluino.
In a specific variant of SUSY, known as R-Parity-Violating or RPV SUSY, the gluinos produced in LHC collisions would
decay to large numbers of Standard Model quarks and gluons .
These events would leave a characteristic signature in the detector, consisting of a large number of high-mass jets.
If gluinos with a low enough mass exist, the rate of these events could be significantly higher than the rate of such events
generated by known Standard Model processes.
The aim of the first half of this thesis is to explain the context and motivation for the analysis.
The second half will consist of explaining the methods used in the analysis, and a discussion of the results.

Chapter~\ref{ch:sm} gives more detail about the Standard Model.
The mathematical formalism, in the form of Quantum Field Theory (QFT) will be described.
The representations of the fields and the Lagrangian governing their dynamics will also be explained.
After a discussion of the phenomenological predictions of the theory, section~\ref{sec:sm_limits} will detail several of the limitations of the Standard Model, which help motivate the need for a BSM theory like SUSY .
Chapter~\ref{ch:susy} details how SUSY can resolve several of the questions left unanswered by the Standard Model.
A brief description of the theory and the particles it predicts is given.
The chapter ends with a discussion of R-Parity violation, and the specific process that is the subject of this search.
Next, chapter~\ref{ch:atlas} provides technical descriptions of the Large Hadron Collider and the ATLAS detector.
The first section of the chapter explains how the LHC accelerates protons, including the magnet systems that guide the beams around the ring, as well as the ones used to focus the beams at the collision points.
The second section details the ATLAS detector, one of the two general-purpose detectors built at the LHC .
ATLAS is comprised of many sub-detectors, each with their own specific purposes, which will be described in detail.
Chapter~\ref{ch:jets} explains the phenomenon of jets in particle collision experiments.
The mechanisms that give rise to jets in proton-proton collisions are first explained, before detailing the experimental methods used to measure jets.
This includes a discussion of jet clustering algorithms that can be used in any particle detector, as well as the specifics of how jets are measured and calibrated in ATLAS in particular.
Chapter~\ref{ch:analysis_overview} introduces the strategy used this particular search.
A description of the signal and its signature in the detector is given, followed by details of the discriminating variables and how they can be used to distinguish signal from background.
Following this, chapter~\ref{ch:data_and_event_selection} describes the data and event selection used for the analysis, as well as the particular choice of jet reconstruction parameters and requirements.
Chapter~\ref{ch:monte_carlo} details the simulated events that were generated for use in the analysis.
Simulated signal events are generated for limit-setting, and simulated background events are generated for validating the data-driven background estimation method.
The data-driven background estimation method, involving jet mass templates, is described in detail in chapter~\ref{ch:background_method}.
A key part of the analysis is the estimation of systematic background uncertainties.
Additionally, the treatment of possible signal contamination is discussed.
The determination systematic and statistical uncertainties on the signal yields is given in~\ref{ch:signal_systematics}.
Finally, chapter~\ref{ch:results} shows the results of the search, including the observed and predicted yields in the signal region.
The statistical interpretation of these results is also given in this chapter.
Conclusions are presented in~\ref{ch:conclusion}.
