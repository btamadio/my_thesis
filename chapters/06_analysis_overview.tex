\chapter{Analysis Overview}\label{ch:analysis_overview}

\section{Strategy Overview}\label{sec:overview}

A data-driven method is used to predict the background yield in the signal regions, as well as the uncertainties on those predictions.
First, jet mass templates are created from control region jets.
Randomized jet masses, known as dressed masses, are generated from these templates for each jet in the kinematic sample.
Summing the dressed masses for each of the up to four leading jets in an event gives the dressed $M_{J}^{\Sigma}$ for that event.
The dressed $M_{J}^{\Sigma}$ distribution for each signal region is used to estimate the expected background contribution to that region.

\section{Data-Driven Background Estimation} \label{sec:bkg_estimation}
A data-driven jet mass template method is used to estimate the background.
The method is similar to that used in the Run-1 version of the analysis~\cite{run1-multijet}, with a few important differences.

In the Run-1 analysis, the templates were smoothed with a kernel density estimate before sampling the dressed masses.
In this version of the analysis, the templates remain binned.
This allows for an estimate of the statistical uncertainty from the control sample size.
By Poisson fluctuating each template bin before sampling, the statistical uncertainty is propagated to the dressed $M_J^{\Sigma}$ distributions.

Secondly, in the Run-1 version of the analysis, two separate sets of templates were generated: one set for the leading two jets in each event,
and a separate set for the third leading jet.
In this analysis, the templates are instead divided into b-matched and non-b-matched jets.
The discrepancy in template shape between b-matched and non-b-matched jets was seen to be larger than that between the third leading jet and first two leading jets.

Finally, the Run-1 version of the analysis used Monte-Carlo non-closure as one contribution to the background systematic uncertainty.
In this analysis, a data-driven method is used instead.
This is due to the fact that the sample size of available simulated data was not large enough to make an accurate estimate of the non-closure.